\section{Introduction}\label{sec:intro}



%% Context
The past 30 years have brought the discovery of roughly 5000 planets beyond our solar system, most detected by the NASA Kepler \citep{borucki2011-Characteristics} and TESS \citep{dragomir2019-TESS, cacciapuoti2022-TESS} missions. Amongst the catalogue of confirmed exoplanets, the majority are larger and more massive than Earth \citep{nasaexoplanetscienceinstitute2020-Planetary}; biases in  detection methods make large, rapidly-orbiting planets easier to detect \citep{burrows2014-Highlights}. Nevertheless, increasing numbers of smaller terrestrial planets have been detected \citep{bryson2020-Occurrence}. The TRAPPIST-1 system \citep{gillon2017-Seven} is perhaps the most publicised due to it hosting a large number of planets, but recently more exoplanets very similar to Earth in size and equilibrium temperature have been detected \citep{gilbert2020-First}. With the development of improved analytical methods and the launch of the latest generation of telescopes, it is now possible to characterise the composition of exoplanetary atmospheres \citep[e.g.,][]{alderson2022-Early}, greatly enhancing the possibility of using climate models to understand these exotic worlds.



%% Clouds and Planets
A key aspect of planetary atmospheres is the presence or absence of clouds. The existence of clouds relies on the presence of a gaseous chemical species in a planet's atmosphere that can condense into cloud particles. Clouds on Earth are a familiar concept, but it is not the only planet where they are found. Beyond Earth, clouds of varying nature have been observed on other terrestrial bodies, both within our solar system and beyond. Our nearest neighbours Venus and Mars both have observable clouds: Venus's take the form of a thick global layer of sulphuric acid \citep{krasnopolsky1981-Chemical}; on Mars, tenuous water ice and carbon dioxide ice clouds have been observed \citep[e.g.,][and references therein]{curran1973-Mars, belliii1996-Detection, haberle2017-Atmosphere}. Saturn's moon Titan hosts a great variety of clouds, with a varying seasonal distribution, morphology, and composition \citep[e.g.,][]{griffith2006-Titan, dekok2014-HCN, turtle2018-Titan}.

Using visible and near-infrared spectra from telescopes such as the Very Large Telescope, the Hubble Space Telescope and the James Webb Space Telescope, we have been able to infer the presence of clouds on exoplanets \citep{kreidberg2014-Clouds, sing2016-Continuum, samland2017-Spectral, barstow2021-Curse}. For instance, \citet{alderson2022-Early} inferred the existence of clouds in the atmosphere of WASP-39b through the continuum transit depth observed with JWST. The sheer variety of exoplanets thus far discovered has also indicated the presence of exotic condensing species, such as metal oxides and silicates in the atmospheres of very hot planets \citep[e.g.,][]{helling2019-Exoplanet}.




%% Impact of Clouds
Clouds have a significant impact on the climate of a planet, both through their influence on the transport of condensable species in the atmosphere and on the radiation budget of the climate system \citep{trenberth2012-Tracking}. Clouds act as a source and sink of moisture, which, via latent heating associated with phase transitions, modify the temperature and stability of their atmospheric environment. Their optical properties as effective scattering media have an impact on both incoming short-wave radiation (affecting planetary albedo) and long-wave radiation emitted by the planet; these cloud radiative feedbacks are broadly understood \citep{webster1994-Role, pierrehumbert2010-Principles} and can be reasonably represented in models \citep{rose2021-Climate}, though clouds also represent one of the larger uncertainties in state-of-the-art models of Earth's climate \citep{ceppi2017-Cloud}. Cloud formation is influenced by a number of factors, chiefly the temperature and specific humidity of an air parcel. These quantities are themselves influenced by processes such as convection and advection, such that large-scale atmospheric dynamics are a significant driver of global cloud distribution.

Despite their importance to the energy budget of the climate system, clouds are often neglected in planetary climate models due to difficulties in modelling them accurately. Clouds present a range of complex microphysical processes that are not fully understood: consider the formation of different ice crystals \citep{storelvmo2015-WegenerBergeronFindeisen}, the physics of mixed-phase clouds \citep{korolev2003-Supersaturation}, or the subsequent formation and growth of raindrops \citep{bergeron1935-Physics}. In addition, the processes and dynamics within clouds occur at small scales, beyond the spatial resolution of general circulation models (GCMs). These sub-grid processes must be parameterised in order to be represented in such models, and there is great variation between cloud schemes of different GCMs. Further, cloud schemes add an additional computational cost to any model. As a result, clouds are often ignored and excluded, even in simplified models of the Earth \citep{thomson2019-Hierarchical}. Nonetheless, given the clear evidence of clouds on other bodies within our own solar system and increasing observations and inferences of exoplanetary clouds, as well as their critical importance to Earth's climate system, it is pertinent to attempt to better account for clouds' impacts in models, especially given their potential importance to observable quantities such as albedo and emission or transmission spectra. 


%% Literature Review
Separately, a number of previous studies have investigated how the atmospheric circulation and climate of a terrestrial planet may be influenced by different atmospheric and planetary characteristics. Planetary rotation rate is well understood as a key determiner of atmospheric dynamics, affecting the width and magnitude of the Hadley circulation \citep{kaspi2015-ATMOSPHERIC, guendelman2018-Axisymmetric, singh2019-Limits, hill2022-Theory}, latitudinal distribution of extratropical eddies \citep{eady1949-Long, taylor1980-Roles}, related extratropical jets \citep{williams1978-Planetary, cho1996-Emergence, chemke2015-Latitudinal}, heat transport\citep{liu2017-effect, cox2021_radiative}, radiative cooling \citep{zhang2023_inhomogeneity}, and general climate and habitability \citep{yang2014-STRONG, haqq-misra2018-Demarcating, komacek2019-Atmospheric, jansen2019-Climates, guzewich2020, cox2021_radiative, he2022-Climate}. The role of rotation rate has often been studied using idealised GCMs, which strip out factors such as topography in order to understand the fundamental physics that drive the dynamics and climate \citep{schneider2006-General, ogorman2008-Hydrological, thomson2019-Effects}. More broadly, \citet{kaspi2015-ATMOSPHERIC} used a number of parameter sweeps to investigate the effects of rotation rate, mean insolation, atmospheric mass, atmospheric optical density, planetary density and radius on the resulting atmospheric circulation in an idealised, Earth-like GCM. Their simulations were based on a clear-sky, equinoctinal aquaplanet with a grey radiation scheme, constructed using the Flexible Modeling System developed at GFDL \citep{held1994-Proposal, anderson2004-New, frierson2006-GrayRadiation}. They found that rotation rate is a key driver of atmospheric dynamics, describing two principal regimes: fast rotators, characterised by a weaker Hadley circulation and development of extratropical jets; and slow rotators, with strong Hadley circulation and a small meridional temperature gradient. \citet{ogorman2008-Hydrological} used a similar grey radiation model to study the connection between the global mean temperature and the hydrological cycle, finding that, as the climate warms, the global-mean precipitation eventually reaches an asymptotic value. Whilst these models included moisture and a representation of the hydrological cycle, all excluded a cloud scheme. 

Various past efforts have used GCM simulations with models of higher complexity to interrogate clouds. For example, \citet{parmentier2016-TRANSITIONS} compared cloud presence in exoplanet transit data of hot Jupiters to models building on the thermal structure from GCMs. They also suggested that cloud presence and composition may be deduced from asymmetries in the light curves of transiting cool exoplanets, providing a potential basis for diagnosing clouds from observational data. \citet{komacek2019-Atmospheric} built on the previous work by \citet{kaspi2015-ATMOSPHERIC}, making use of the \textsc{ExoCAM} model \citep{wolf2022-ExoCAM}, to perform a range of GCM simulations including sea ice, a correlated-k radiation scheme, and a cloud scheme \citep{rasch1998-Comparison}. Their findings were qualitatively consistent with the results of \citet{kaspi2015-ATMOSPHERIC}, but they observed a larger equator-to-pole temperature gradient. This may be attributed to their use of a more complex radiative scheme and addition of a cloud scheme; they also noted that cloud particle size appears to be an important but uncertain parameter. While they presented an interesting transition between low and high day-side cloud coverage for synchronously rotating planets with varying rotation periods, the impact of clouds on atmospheric dynamics and climate for asynchronous planets was not explicitly addressed. \citet{yang2013-STABILIZING} and \citet{yang2014-STRONG} also investigated clouds over a range of rotation rates, including tidally-locked systems. They suggested that the planetary albedo increases to a maximum in the synchronous case, where a substantial cloud cap forms at the sub-stellar point. Finally, \citet{guzewich2020} showed that variations of cloud properties with rotation rate can alter the observable signatures of otherwise Earth-like planets in reflected light and thermal emission spectra.

Most of the previous studies generally considered planets without a seasonal cycle, hence providing a hemispherically symmetric circulation. Whilst simulating a planet with no obliquity ($\varepsilon=0$) has advantages, we know from our own solar system that planets rarely conform to this simplification, as Earth, Mars and Titan all have large effective obliquities of approximately $\varepsilon=25^\circ$. This results in seasonally-varying radiative forcing that is a key driver of dynamics that are absent in aseasonal climates \citep{guendelman2018-Axisymmetric, guendelman2019-Atmospheric, ohno2019_atmospheres, singh2019-Limits, hill2022-Theory}, including the Earth's monsoon systems \citep{bordoni2008-Monsoons, geen2018-Regime}; in this context, for example, the migration of the Intertropical Convergence Zone (ITCZ) has been shown to vary non-monotonically with rotation rate \citep{faulk2017-Effects,geen2018-Regime}. Other studies have explored how obliquity influences the climatological temperature, precipitation and habitability \citep{ferreira2014-Climate, kang2019-Mechanisms, kodama2022-Climate, linsenmeier2015-Climate, lobo2020-Atmospheric, he2022-Climate}. Furthermore, \citet{lobo2022-Role} analysed the relationship between extratropical storminess and longwave radiation, and \citet{hadas2023-Role} took cloud albedo into consideration, linking Earth's large-scale circulation to planetary albedo, which indicates the importance of using a more realistic radiative scheme to study the effect of rotation rate.



This work builds on the above studies, especially those by \citet{kaspi2015-ATMOSPHERIC} and \citet{komacek2019-Atmospheric}, by interrogating the effects of clouds on varying planetary climates with idealised GCM simulations including a seasonal cycle. By covering a large range of rotation rates, we aim to include a swathe of dynamical regimes with a likely impact on cloud distributions, enabling us to investigate the response of seasonal cloud behaviour to rotation rate and the feedback of these clouds on the seasonal climate. We do not consider tidally locked systems where rotation and orbital rate are synchronous; tidally locked planets display a distinct dynamical regime \citep{joshi1997-Simulations, noda2017-Circulation, haqq-misra2018-Demarcating, pierrehumbert2019-Atmospheric, wordsworth2022-Atmospheres} where dynamics are not longitudinally invariant \citep{merlis2010-Atmospheric, sergeev2020-Atmospheric, hammond2021-Rotational}, and the presence and influence of clouds on tidally locked planets have already been investigated extensively \citep{yang2013-STABILIZING, yang2019-Simulations, helling2021-Clouds, sergeev2022-TRAPPIST1}.


Hereon, we describe the experimental setup and procedure in Section 2. Following this we present our results, starting with the effect of rotation rate on the dynamics and large-scale distribution of clouds, including their impacts on planetary albedo (Section 3.1). We follow this with results investigating the consequences of clouds upon climate and seasonality (Section 3.2), and discuss comparisons with previous work and provide a future outlook in Section 4.