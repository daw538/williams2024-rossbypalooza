\section{Methods}\label{sec:methods}


\subsection{Idealised GCM Setup}
In order to explore the effect of planetary rotation rate on the dynamics of a simplified cloudy aquaplanet, we make use of the flexible modelling framework Isca \citep{vallis2018-Isca}. Isca is not a single model, but a framework based on the GFDL Flexible Modeling System for constructing a range of idealised GCMs covering a wide hierarchy of complexity \citep{thomson2019-Hierarchical}, allowing the effects of particular processes to be studied in isolation. Complexity may range from a very simple Newtonian thermal-relaxation model as described by \citet{held1994-Proposal} to models that introduce additional complexity such as moist physics, topography and differing radiation schemes. 

The initial base for our setup in Isca can be derived from the moderately complex Earth-like model described in \citet{thomson2019-Hierarchical}. We use a 360-day calendar with all planetary parameters identical to Earth except rotation rate $\Omega$. We also use a simplified orbit, including the Earth's obliquity but neglecting eccentricity such that seasonal forcing is symmetrical across the northern/southern hemispheres.

Isca uses a spectral dynamical core in spherical coordinates to solve the primitive equations. We use a spectral resolution of T42 for most experiments, providing a grid of 64 latitude and 128 longitude cells, each of approximately \ang{2.8}$\times$\ang{2.8} in size. For higher rotation rate simulations, the decreasing Rossby deformation scale that results from a higher value of $\Omega$ demands a higher spectral resolution of T85 \citep{kaspi2015-ATMOSPHERIC, vallis2017-Atmospheric}. Each model has a vertical grid defined by 50 unevenly spaced levels up to a pressure height of \SI{0.02}{\hecto\Pa}. 

We specify a global mixed-layer slab ocean of \SI{2.5}{\m} depth without prescribed surface Q-fluxes \citep{jucker2017-Untangling}. Isca does not include a dynamical ocean model, so this allows for a closed surface energy budget with the mixed layer depth setting the thermal inertia of the slab ocean; the value is low to allow a strong response to seasonal forcing \citep{donohoe2014-Effect, jucker2019-SURFACE, liu2021-SimCloud}. An albedo of $0.2$ is used to account for the additional albedo that clouds provide \citep{liu2021-SimCloud}; the value is used in all experiments for consistency across the parameter sweep.

To represent moist convection, we use the Simple Betts-Miller (SBM) scheme described by \citet{frierson2007-Dynamics}. SBM is a quasi-equilibrium scheme \citep{betts1986-New, betts1986-Newb} that relaxes to a relative humidity profile of \qty{70}{\percent} and temperatures to a moist adiabat over a \SI{7200}{\s} timescale when convective instability occurs. This is coupled with a large-scale condensation scheme that removes excess vapour beyond saturation in each grid cell. We use \textsc{SOCRATES} \citep{edwards1996-Studies, manners2016-SOCRATES} for the radiation scheme, which provides higher complexity relative to simpler grey-radiation schemes: \textsc{SOCRATES} is a comprehensive, multi-band scheme developed by the UK Met Office which has already been used in applications beyond present-day Earth \citep{amundsen2016-UK, way2017-Resolving}. We run \textsc{SOCRATES} with the included \texttt{ga7} spectral files, providing 9 long-wave and 6 shortwave bands (standard configuration for an Earth-like model), and making use of stratospheric ozone absorption to provide a closer comparison with Earth's atmospheric temperature structure.



\subsection{Cloud Scheme and Rotation Rates}
To implement clouds in the simulations, we make use of the SimCloud scheme developed by \citet{liu2021-SimCloud}. In general, cloud schemes add significant computational overhead to models since the physics of clouds are complex, necessitating interface with multiple parts of a GCM. SimCloud aims to provide a simple diagnostic cloud scheme for examining the impacts of clouds on climate in idealised models. SimCloud diagnoses clouds by local environmental variables and only interacts with the radiation scheme. Simplifications include treating water and ice clouds both as liquid but with different effective cloud particle radii as derived from observations \citep{stubenrauch2013-Assessment}. Large-scale clouds are diagnosed using the local grid-mean relative humidity, with a freeze-dry adjustment applied to prevent the overestimation of polar clouds. Marine low-level clouds such as stratocumulus are diagnosed using the local vertical temperature profile. We employ the following options within the scheme for cloud diagnostics \citep[identical to runs from][]{liu2021-SimCloud}: a linear large-scale cloud diagnostic formula; maximum-random assumption for cloud overlap; and the Park-ELF method for low-level marine clouds. Finally, precipitation in Isca is idealised and decoupled from the cloud scheme, being driven by either large-scale condensation \citep{frierson2006-GrayRadiation} or humidity relaxation from the SBM convection scheme \citep{frierson2007-Dynamics}. This allows precipitation formation in models with or without a cloud scheme present.


We present simulations with the model using 22 different rotation rates. With Earth's rotation rate labelled $\Omega_E$, we define $\Omgstar = \Omega/\Omega_E$, which is the ratio of the planetary rotation rate compared to Earth's. $\Omgstar$ takes values between $1/128$ and $4$, extending from very slow to fast rotators, and encompassing the rotation rates of Earth, Mars and Titan. At each rotation rate, we use configurations of the model with the cloud scheme enabled and disabled to provide a comparison with a clear-sky simulation, allowing the impacts from cloud forcing to be more clearly identified. Each simulation lasts for 15 Earth years, sufficient for the large-scale dynamics of the troposphere to reach a steady-state; we discard the initial 10 as a spin-up period, and use the final five years to provide climatological averages for each experiment.