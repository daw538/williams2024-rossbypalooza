\section{Summary and Discussion}\label{sec:discussion}

Using a suite of simulations with an idealised terrestrial aquaplanet GCM, we demonstrate the relationship between rotation rate, cloud distribution and planetary albedo under the influence of seasonal forcing. We show that rotation rate influences the cloud distribution via the large-scale circulation; consequentially, cloud feedbacks impact the seasonality of the circulation itself. Our key findings are summarised as follows:
\begin{enumerate}
    \item The seasonal cycle leads to a dramatic shift in cloud distribution with rotation rate. At slow rotation rates, cloud area fraction displays a dipolar behaviour near the solstices, as the summer hemisphere is very cloudy whilst the winter hemisphere is relatively cloud-free. At high rotation rates, the clearest distinction between seasons occurs near the equinox rather than the solstice, with the autumnal hemisphere significantly cloudier than the spring hemisphere. The latitudinal distribution of clouds at high rotation rate is more complex with the development of separate equatorial and mid-latitude regions of cloudiness.
    
    \item Cloud distribution is well explained by the large-scale circulation, with the cloudiest regions associated with those undergoing ascent within the atmosphere and having a small (or negative) gradient in equivalent potential temperature. Broadly speaking, low-level clouds occur with a more global distribution, more strongly influenced by the boundary layer.
    
    \item The seasonal cloud distribution greatly impacts the planetary albedo as a function of rotation rate, with substantial inter-seasonal variability. Planetary albedo displays a non-monotonic behaviour that peaks at around $\Omgstar\approx 1/2$, decreasing towards slower rotations as the ITCZ and peak cloud distribution move poleward, and likewise towards faster rotation rates as eddy length scale decreases, reducing the scale of extratropical cyclonic systems that contribute to cloud formation. The local minimum in albedo at $\Omgstar\approx 1/8$ is likely linked to the transition point between a circulation that is mean-flow dominated at slow rotation rates to eddy dominated at high rotation rates.

    \item The introduction of clouds leads to a decrease in annual global mean precipitation compared to cloud-free simulations at all rotation rates, related to their impact on the energy budget. Without clouds, precipitation exhibits a monotonic increase with increasing rotation rates. Once clouds are introduced, a peak in precipitation develops around $\Omgstar\approx 1/8$. Precipitation remains relatively constant for rotation rates below $\Omgstar<1/8$ and increases with rotation rates above $\Omgstar>1/4$.

    \item At slow rotation rates ($\Omgstar<1/8$) the inner edge of the winter Hadley cell and the ITCZ are found near \ang{90} latitude, with little distinction between cloudy and cloud-free cases. At higher rotation rates, a significant difference between the two cases develops, which changes the sign at the fastest rotation rates.

    \item Increasing rotation rate increases the magnitude of the seasonal phase lag in both cloudy and cloud-free simulations. The addition of clouds reduces the seasonal phase lag.

\end{enumerate}

Our results cover a large range of rotation rates, from slow rotators such as Titan to fast rotators such as Earth. This covers a similar parameter space to the simulation grid performed by \citet{kaspi2015-ATMOSPHERIC}, but we do not include simulations equivalent to their fastest ($\Omgstar>4$). This is justified on practical and scientific bases: \citet{chemke2015-Poleward} demonstrated that for fast rotating Earth-like systems, the eddy-driven jets migrate latitudinally even in the absence of seasonal forcing, an effect that does not occur at slower rotation rates. This would be incompatible with any analysis of seasonal averages, particularly with the link demonstrated between large-scale circulation and cloud distribution---we note that our fastest rotating models $\Omgstar \gtrsim 3$ may be affected by this. Running models of higher rotation rates also necessitates the use of higher spectral resolutions and lower model timesteps, imposing a significant penalty on model run-time. At the other end, \citet{yang2014-STRONG} found that slowly rotating planets exhibit strong convergence and convection in the substellar region, leading to the formation of extensive optically thick clouds and a significant increase in planetary albedo. Their results show an increase in planetary albedo with decreasing rotation rates when $\Omgstar<1/8$. However, in our simulations, we did not investigate the tidally-locked atmospheric dynamical regime, even at extremely low rotation rates such as $\Omgstar=1/128$. Our simulations did not exhibit a day-night contrast but rather revealed zonal banded patterns in the daily mean (24-hour mean) climatology.

Regarding how seasonality varies with rotation rate, \cite{faulk2017-Effects} used a more idealised moist GCM, also based on the GFDL spectral dynamical core, to investigate the dependence of ITCZ migration on rotation rate. Their model did not include the effects of clouds and used an idealised gray radiative transfer scheme. As discussed in Section \ref{sec:results-seaonality}, our results generally agree with theirs, except for the maximum latitude of the ITCZ and Hadley cell extent for cases with $\Omgstar<1/8$. \cite{faulk2017-Effects} showed that the ITCZ remains approximately at $\sim\,$\ang{60} while the Hadley cell extent remains at \ang{75}--\ang{80}, while, in our simulations, both the ITCZ and Hadley cell migrate to \ang{90} in cases with and without clouds (Figure \ref{fig:ITCZ} \& \ref{fig:lat-rotation}). We speculate that the difference in the radiation scheme, which leads to insolation and longwave differences, may help determine the location of the ITCZ, which suggests important model dependence. Further, our use of a low mixed-layer depth for the slab ocean allowed the ITCZ to migrate farther poleward in our simulations. Separately, \cite{guendelman2022-Key} and \cite{tan2022-Weak} investigated how the phase lag of seasonal transition varies with rotation rate and concluded that, with a slower rotation rate, the increase in meridional heat transport would decrease the phase lag, which is also seen in our simulations (Figure \ref{fig:Seasonal_Transition} \& \ref{fig:Seasons_Effect-contour}).

We compare our results of simulations with clouds to the results of \citet{komacek2019-Atmospheric}. Komacek \& Abbot used a GCM of higher complexity \citep[ExoCAM;][]{wolf2022-ExoCAM} that includes a correlated-$k$ radiation scheme, allows sea ice to form, and uses sub-grid parameterisations for clouds from \cite{rasch1998-Comparison}. However, their simulations assumed zero obliquity and no seasonal effects. Therefore, the absolute values of streamfunction and jet speed in their results are different from ours. To account for this, we normalise the values to Earth's rotation rate, and we find similar trends: the maximum streamfunction decreases with rotation rate, and the maximum jet speed peaks at around $\Omgstar=1/8$. Our results regarding the latitude of the edge of the Hadley cell and the jet also overlap with those of \citet{komacek2019-Atmospheric}. However, our results on how cloud fraction changes with rotation rate do not agree. In ExoCAM simulations, cloud fraction overall decreases with rotation rate, while our Isca simulations show an increasing trend. There is no clear albedo kink occurring around $\Omgstar=1/8$ in \citet{komacek2019-Atmospheric}, possibly due to their sparse data points on rotation rate but also likely as a consequence of the aseasonality and model configuration differences. The intrinsic difference in cloud parameterisation between the two models, in particular, warrants further research. 

It is likely that there may be differences between our and others' results simply through the use of a particular cloud scheme; \citet{komacek2019-Atmospheric} noted that their results may differ from GCMs using alternative cloud parameterisations. Each cloud scheme is often tuned to reasonably represent the distribution of clouds on present-day Earth. We acknowledge that our simulations present a novel use of the SimCloud scheme, differing primarily from the work by \citet{liu2021-SimCloud} in using an aquaplanet setup and altering the rotation rate. To our understanding, however, none of our changes should invalidate the simple cloud parameterisation itself. Indeed, our study aims at a fundamental understanding of the distribution of clouds in terms of key planetary parameters enabled by a more idealised representation without the complications introduced by effects, such as land-sea heat capacity gradients and orographic uplift, that influence cloud formation.

As well as helping to elucidate the impact of clouds in terrestrial planetary climate systems, our results may be of use to the observation and characterisation of exoplanets. Proposed missions such as the Habitable Worlds Observatory (HWO) aim to study Earth-like exoplanets in reflected light via direct imaging methods. If an Earth-like planet displays an inhomogenous distribution of reflected light (albedo) across the planetary disk, it may be indicative of a cloud-bearing atmosphere \citep{cowan2009-ALIEN, lustig-yaeger2018-Detecting}. With sufficiently long monitoring, it may be possible to infer the rotation rate of the planet by studying changes in the albedo distribution, whose distribution and seasonality we have shown to be related to rotation rate. Regardless, there is ample scope for relating this theoretical work to future observational campaigns.